\section{Problematização}

É fato que a sociedade possui grande sede de consumo, principalmente voltada a indústria alimentícia, o evidente crescimento populacional agrava a geração de resíduos sólidos que contenham produtos que buscam saciar tais necessidades. Um fator alarmente em nível global foi a falta do processo de conscientização populacional sobre os efeitos causados caso as empresas e as pessoas não tomem parte responsável sobre aquilo que produzem e consomem, fato é que, a natureza sofre bastante com consequências advindas deste cenário.

Buscando um meio de contornar tal situação, entusiastas do meio ambiente e governos conscientizados geraram alguns projetos com intuito de minizar e controlar os danos a natureza, um deles que ficou em evidência é máquina automática de reciclagem, produto/protótipo que será desenvolvido neste projeto.

\section{Objetivo Geral}

\subsection{Problema}
Com base no que esta contido na descrição anterior, é cabível concluir que os problemas principais são: a alta produção de resíduos sólidos e que tais são jogadas sem pudor na natureza e a falta de interesse de várias pessoas por tal causa.  

\subsection{Solução}
Buscando ajudar empresas de reciclagem e inserir mais pessoas a este tipo de ação, este projeto tem como objetivo geral a construção de um protótipo exemplar de uma máquina que automatiza o recolhimento de garrafas plásticas e de vidro por meio da técnica de bonificação para as pessoas. Serão utilizados os conhecimentos em conjunto as 5 áreas de engenharia presente no campus do gama da Universidade de Brasília, onde as áreas de Aeroespacial e Automativa ficarão responsáveis pela estrutura, a área de Energia pelo controle energético e de segurança, a área de Eletrônica pela automação e controle eletrônico e a área de Software pela interação usuário máquina e planejamento geral.

\section{Objetivos Específicos}

\subsection{Problema na Visão das Engenharias}
O problema apresentado pode ser visto voltado separadamente para cada engenharia em visões mais técnicas, no caso, tomando as áreas de Automotiva e de Aeroespacial como uma em Estrutura. Falando então em âmbito de estrutura, esta cabe ter a visão de que é complexo e trabalhoso a construção de um sistema que controla e armazena de formas diferentes, diferentes tipos de materiais. Na visão eletrônica, hoje em dia, vários processos de separação e validação de objetos reciclavéis são realizados de forma manual. Na visão de energia, é comum encontrar sistema de segurança e controle energético falhos. E na visão de \textit{Software}, é complicado manter as pessoas integradas com tais ações (tanto a integração da equipe em relação a planejamento, quanto manter os usuários utilizando novos sistemas para as causas ambientais já citadas).

\subsection{Soluções na Visão das Engenharias}
\begin{itemize}
\item Construção de estrutura que suporte os componentes de funcionamento da máquina
\item Implantação de sistema de segurança
\item Implantação de sistemas de controle de energia
\item Implementação de sistema de separação e armazenamento dos materiais
\item Validação de objetos inseridos na máquina
\item Controle automático dos dados do usuário e sua interação com a máquina
\end{itemize}